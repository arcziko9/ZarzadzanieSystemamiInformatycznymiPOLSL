%%%%%%%%%%%%  Generated using docx2latex.com  %%%%%%%%%%%%%%

%%%%%%%%%%%%  v2.0.0-beta  %%%%%%%%%%%%%%

\documentclass[12pt]{article}
\usepackage{amsmath}
\usepackage{latexsym}
\usepackage{amsfonts}
\usepackage[normalem]{ulem}
\usepackage{soul}
\usepackage{array}
\usepackage{amssymb}
\usepackage{extarrows}
\usepackage{graphicx}
\usepackage[backend=biber,
style=numeric,
sorting=none,
isbn=false,
doi=false,
url=false,
]{biblatex}\addbibresource{bibliography.bib}

\usepackage{subfig}
\usepackage{wrapfig}
\usepackage{wasysym}
\usepackage{enumitem}
\usepackage{adjustbox}
\usepackage{ragged2e}
\usepackage[svgnames,table]{xcolor}
\usepackage{tikz}
\usepackage{longtable}
\usepackage{changepage}
\usepackage{setspace}
\usepackage{hhline}
\usepackage{multicol}
\usepackage{tabto}
\usepackage{float}
\usepackage{multirow}
\usepackage{makecell}
\usepackage{fancyhdr}
\usepackage[toc,page]{appendix}
\usepackage[hidelinks]{hyperref}
\usetikzlibrary{shapes.symbols,shapes.geometric,shadows,arrows.meta}
\tikzset{>={Latex[width=1.5mm,length=2mm]}}
\usepackage{flowchart}\usepackage[paperheight=11.69in,paperwidth=8.27in,left=1.0in,right=1.0in,top=1.0in,bottom=1.0in,headheight=1in]{geometry}
\usepackage[utf8]{inputenc}
\usepackage[T1]{fontenc}
\TabPositions{0.5in,1.0in,1.5in,2.0in,2.5in,3.0in,3.5in,4.0in,4.5in,5.0in,5.5in,6.0in,}

\urlstyle{same}

\renewcommand{\_}{\kern-1.5pt\textunderscore\kern-1.5pt}

 %%%%%%%%%%%%  Set Depths for Sections  %%%%%%%%%%%%%%

% 1) Section
% 1.1) SubSection
% 1.1.1) SubSubSection
% 1.1.1.1) Paragraph
% 1.1.1.1.1) Subparagraph


\setcounter{tocdepth}{5}
\setcounter{secnumdepth}{5}


 %%%%%%%%%%%%  Set Depths for Nested Lists created by \begin{enumerate}  %%%%%%%%%%%%%%


\setlistdepth{9}
\renewlist{enumerate}{enumerate}{9}
		\setlist[enumerate,1]{label=\arabic*)}
		\setlist[enumerate,2]{label=\alph*)}
		\setlist[enumerate,3]{label=(\roman*)}
		\setlist[enumerate,4]{label=(\arabic*)}
		\setlist[enumerate,5]{label=(\Alph*)}
		\setlist[enumerate,6]{label=(\Roman*)}
		\setlist[enumerate,7]{label=\arabic*}
		\setlist[enumerate,8]{label=\alph*}
		\setlist[enumerate,9]{label=\roman*}

\renewlist{itemize}{itemize}{9}
		\setlist[itemize]{label=$\cdot$}
		\setlist[itemize,1]{label=\textbullet}
		\setlist[itemize,2]{label=$\circ$}
		\setlist[itemize,3]{label=$\ast$}
		\setlist[itemize,4]{label=$\dagger$}
		\setlist[itemize,5]{label=$\triangleright$}
		\setlist[itemize,6]{label=$\bigstar$}
		\setlist[itemize,7]{label=$\blacklozenge$}
		\setlist[itemize,8]{label=$\prime$}

\setlength{\topsep}{0pt}\setlength{\parskip}{8.04pt}
\setlength{\parindent}{0pt}

 %%%%%%%%%%%%  This sets linespacing (verticle gap between Lines) Default=1 %%%%%%%%%%%%%%


\renewcommand{\arraystretch}{1.3}


%%%%%%%%%%%%%%%%%%%% Document code starts here %%%%%%%%%%%%%%%%%%%%



\begin{document}
\textbf{Dokument roboczy projekt ZSI Moduł 3}\par


\vspace{\baselineskip}
\textbf{Arkadiusz Kałuża, Adam Kierat}\par


\vspace{\baselineskip}
Przygotowanie dokumentu roboczego zawierającego zapisy dotyczące punktów wyróżnionych we wzorze dokumentu roboczego dostęnego na platformie zdalnej edukacji.\par


\vspace{\baselineskip}
\begin{itemize}
	\item \textbf{A1. Ustanowienie współdzielonego repozytorium}\\
Ustanowienie\ współdzielonego\ repozytorium\ na\ platformie\ github.com\ \ \ \ \ \ \ \ \ \ \ \ \ \ \ \       Link: \href{https://github.com/arcziko9/ZarzadzanieSystemamiInformatycznymiPOLSL}{\textcolor[HTML]{1155CC}{\ul{https://github.com/arcziko9/ZarzadzanieSystemamiInformatycznymiPOLSL}}}\\
\par

	\item \textbf{B1. Ustalenie z prowadzącym tematu projektu.}\\
Sekcja wybrała projekt numer 3.8:\\
 \tab a. Przedstaw możliwości wybranego rozwiązania wspomagającego \tab \tab \ \ \  zarządzanie systemami informatycznymi (docker)\\
 \tab b.\ Przedstaw\ możliwości\ tworzenia\ dokumentacji technicznej (podręcznika     \tab \ \ \  użytkownika)\\
 \tab c. Opracuj film demonstrujący silne i słabe strony zadanego systemu \tab \tab \ \ \  informatycznego będący alternatywą dla Asana\\
\par

	\item \textbf{B3. Przygotowanie notatki z ASD: 3 pozycje, 3 najważniejsze wyróżniki, linki do witryn zawierających szczegóły.}\par

\begin{itemize}
	\item a. Platforma Docker jest projektem typu open source do automatyzacji wdrażania aplikacji jako przenośnych, samowystarczalnych kontenerów, które można udostępniać w chmurze lub lokalnie. Doker jest również firmą, która promuje i rozwija tę technologię współpracując z dostawcami chmury, systemu Linux i Windows.\par

\begin{itemize}
	\item https://ujeb.se/DfGJR\par

	\item https://ujeb.se/hZb38\par

	\item https://ujeb.se/xX5Nx\par


\end{itemize}
	\item b. Dokumentacja techniczna powinna być jak instrukcja obsługi – konkretna, czytelna, rozwiewająca wątpliwości, pokazująca użytkownikowi krok po kroku przebieg danej czynności. Podstawowym krokiemi w tworzeniu dokumentacji jest zdefiniowanie kto ma być odbiorcą dokumentacji, żeby odpowiednio dostosować styl pisania jej. Drugim ważnym krokiem jest stworzenie spisu treści w której łatwo będzie można znaleźć odpowiedź na problem, który się natkneliśmy.\par

\begin{itemize}
	\item \href{https://ujeb.se/GQ8Hg}{https://ujeb.se/GQ8Hg} \par

	\item \href{https://ujeb.se/jg0pv}{https://ujeb.se/jg0pv}\par

	\item \href{https://ujeb.se/XLuSo}{https://ujeb.se/XLuSo} \par


\end{itemize}
	\item c. Asana jest do serwis do zarządzania projektami. Świetnie sprawdzi się zarówno w małych, jednoosobowych projektach, jak np. blog, kanał na YouTube, ale także w pracy, przy tych $``$pełnoprawnych$"$ , wieloosobowych, zawodowych. Alternatywami dla Asany są np.WorkZone,Trello,Jira:\par

\begin{itemize}
	\item \href{https://www.workzone.com/%20}{https://www.workzone.com/} \par

	\item \href{https://trello.com/}{https://trello.com/}\par

	\item \href{https://www.atlassian.com/software/jira}{https://www.atlassian.com/software/jira}
\end{itemize}
\end{itemize}
\end{itemize}\par


\vspace{\baselineskip}
\\

\vspace{\baselineskip}\begin{itemize}
	\item \textbf{C1.Opracowanie koncepcji realizacji projektu.}\par

\begin{itemize}
	\item Docker to tak naprawdę zbiór funkcjonalnych narzędzi, który pozwala na uruchamianie wielu różnych procesów w tzw. kontenerach, czyli w odizolowanych środowiskach. To coraz popularniejsze rozwiązanie, które często można spotkać w chmurze. Jego główną zaletą jest to, że może zastąpić wirtualną maszyne. Wygoda jego użytkownika wynika z tego, że kontener pozwala na uruchomienie w sposób prostszy niż wirtualna maszyna – nie ma tu żadnego emulowania warstwy sprzętowej w dodatku nie potrzebujemy systemu operacyjnego.\par

\begin{itemize}
	\item \textbf{Dystrybucja aplikacji\\
}Docker jest tak sprytny, że pozwala wykorzystać obrazy nie tylko do łatwego tworzenia środowiska deweloperskiego. Można to narzędzie użyć także do dystrybucji aplikacji. O ile w przypadku instalacji na serwerze, administrator lub użytkownik aplikacji musi sam poradzić sobie z tym procesem i ze wszystkimi związanymi z tym aspektami (np. inaczej instaluje się oprogramowanie w Javie i inaczej w Pythonie lub Ruby), o tyle wykorzystanie Dockera ułatwia całą sprawę.\\
\\
Programista może łatwo przygotować paczkę dystrybucyjną. Wystarczy, że wykona obraz kontenera z aplikacją, która została ukończona. Już nie administrator czy użytkownik dokonuje instalacji, ale deweloper. Natomiast ci pierwsi dostają kontener, który jest gotowy do uruchomienia i nie muszą zastanawiać się, jaka to technologia, jak ją zainstalować i czy posiadają do tego odpowiednią wiedzę.\par

	\item \textbf{Proces zamknięty w kontenerze\\
}Uruchomienie aplikacji w sposób opisany powyżej jest możliwe, ponieważ Docker pozwala na funkcjonowanie samego procesu aplikacji. W ten sposób można inicjować działanie dowolnej liczby procesów w różnych kontenerach, a każdy z nich posiada przydzielony specjalnie dla niego obszar pamięci, własny adres prywatny IP i interfejs oraz ma własny obszar na dysku, gdzie wcześniej zainstalowano obraz systemu operacyjnego i wszystkich bibliotek, które konieczne są do uruchomienia aplikacji. Ważne jest także to, że każdy z tych kontenerów działa zupełnie niezależnie, ale możemy również tworzyć między nimi połączenia sieciowe. Warto jednak pamiętać, że Docker działa pod kontrolą GNU/Linux, natomiast użytkownicy Windows lub Mac muszą skorzystać z menedżera maszyn wirtualnych VirtualBox. Inne rozwiązanie to używanie maszyny wirtualnej z preinstalowanym Dockerem. To oferuje już coraz więcej dostawców cloud computing (np. Azure, Amazon AWS itd.).\par

	\item \textbf{Wydajność\\
}Z powodu izolowanych kontenerów Docker pozwala administatorom i programistom dostarać szybko działające, centralnie zarządzane i bezpieczne aplikacje rozproszone. Wydajność ta jest jeszcze większa gdy zastosujemy chmure.\\
\par


\end{itemize}
	\item Dokumentacja jest potrzebna. Jest częścią produktu, ma wartość biznesową\textbf{ }- pozwala ograniczyć koszty i pozyskiwać nowych klientów.  W zależności od zastosowania, odbiorcy czy skali projektu, może ona przybierać różne formy. Zazwyczaj pisanie dokumentacji nie wymaga tworzenia tysięcy dokumentów i skomplikowanego języka - treści powinny być proste i zrozumiałe dla jak największej grupy odbiorców. W idealnej sytuacji, dokumentacja może być częścią procesu tworzenia oprogramowania, co pozwala ograniczyć czas potrzebny na jej przygotowanie i publikację.\\
powinniśmy zasięgnąć trochę wiedzy na temat wypracowanych metod skutecznej komunikacji. Dobrym przykładem będzie \textbf{plain language} - czyli zestaw wytycznych, które mówią wprost, co to znaczy, że tekst jest prosty. Celem jest tworzenie dokumentów w taki sposób, aby były zrozumiałe dla możliwie szerokiego spektrum odbiorców.\par

\begin{itemize}
	\item \textbf{Przykładowe narzędzia do tworzenia dokumentacji technicznej:}\par

\begin{itemize}
	\item \textbf{Natural Docs ( \href{https://www.naturaldocs.org/}{https://www.naturaldocs.org/} )\\
}pozwala dokumentować kod napisany w jednym z 21 języków programowania, a ponadto można go łatwo rozszerzyć o więcej, więc bez względu na to, czego używasz, może również. A jeśli twój projekt używa wielu języków, nie ma problemu! Wszystko to będzie zawarte w tym samym zestawie dokumentacji.\par

	\item \textbf{DoxyGen ( \href{http://www.doxygen.nl/}{http://www.doxygen.nl/} )\\
}Ndo generowania dokumentacji ze źródeł C ++ z adnotacjami, ale obsługuje także inne popularne języki programowania, takie jak C, Objective-C, C $\#$ , PHP, Java, Python, IDL (smaki Corba, Microsoft i UNO / OpenOffice) ),\par


\vspace{\baselineskip}

\end{itemize}
\end{itemize}
	\item Opisywaną alterynatywą do Asana będzie aplikacja o nazwie Trello.
\end{itemize}
\end{itemize}\par

\begin{adjustwidth}{1.0in}{0.0in}
Trello umożliwia wizualne zarządzanie notatkami, czy raczej jak mówią twórcy — „wszystkim$"$  i dzielenie się tym ze „wszystkimi$"$ . Nieważne czy pracujesz solo, czy z zespołem, Trello ma tę cudowną zaletę, że jest typowym narzędziem. To, co z nim zrobisz zależy tylko od Ciebie oraz Twoich współpracowników.\par

\end{adjustwidth}

\begin{adjustwidth}{1.0in}{0.0in}
Przykładowe zalety trello:\par

\end{adjustwidth}

\begin{itemize}
	\item  Codzienna organizacja - jeśli Twój dzień jest wypełniony różnego rodzaju zadaniami, posiadanie wszystkich tych zadań w jednym miejscu może być niezbędne. Trello może zestawiać wszystkie twoje zadania w listy i nadal pozwala ci uzyskać doskonały widok na cały dzień.\par

	\item Śledzenie czasu - Śledzenie czasu niekoniecznie musi być tym, które ładuje się z godziny. To naprawdę wspaniały sposób na zwiększenie wydajności.\par

	\item Czytelna lista zadań - Jeśli masz powoli rosnący katalog książek, które chcesz przeczytać, możesz po prostu przechowywać listę w Trello, tworząc karty dla każdej z nich.\par

	\item Łatwe palnowanie wakacji - Planowanie wakacji zazwyczaj wymaga wielu decyzji. Trello pomaga to uprościć, tworząc listy codziennych czynności.\par

	\item  Zarządzaj budżetem - Trello może pomóc w stworzeniu planu finansowego. Możesz skonfigurować karty i śledzić dochody i wydatki, kiedy ich potrzebujesz.\par

	\item Zarządzanie projektami - niezależnie od projektu, Trello będzie do Twojej dyspozycji. Utworzenie tablicy z indywidualnymi listami może zarządzać wszystkimi małymi zadaniami, które są częścią twojego ogólnego projektu.decyzji. Trello pomaga to uprościć, łącza listy codziennych operacji.
\end{itemize}\par


\vspace{\baselineskip}
\begin{adjustwidth}{1.25in}{0.0in}
Wady aplikacji Trello:\par

\end{adjustwidth}

\begin{itemize}
	\item Nie ma wsparcja offline - głównym problemem nie tylko Trello, ale ogólnie tego rodzaju oprogramowania jest poleganie na danych. Jeśli nie masz dostępu do Internetu, nie masz dostępu do swojego Trello. Dostęp do danych może nie wydawać się problemem, biorąc pod uwagę wiek, w którym żyjemy, ale niestety zawsze będą sytuacje, w których nie będzie można uzyskać dostępu do danych.\par

	\item  Pamięć jest ograniczona - chociaż Trello może przechowywać wiele załączników, załączniki te są ograniczone do 250 MB na przesyłanie, jeśli jesteś złotym członkiem. Problem polega na tym, że masz tylko 10 MB na przesyłkę, jeśli jesteś podstawowym członkiem.\par

	\item Komentowanie - ciągłym problemem związanym z Trello, który należy naprawić, jest to, że nie można edytować komentarza na karcie. Po opublikowaniu i zapisaniu komentarza pojawi się opcja napisania nowego komentarza zamiast edycji oryginalnego.\par

	\item Zarządzanie dużymi projekatmi bywa problematyczne - jeśli chodzi o mniejsze projekty, Trello ma swoje. Jeśli jednak zarządzanie projektami na większą skalę jest wymagane, być może Trello nie jest najlepszą opcją.
\end{itemize}\par


\vspace{\baselineskip}

\vspace{\baselineskip}

\vspace{\baselineskip}

\printbibliography
\end{document}