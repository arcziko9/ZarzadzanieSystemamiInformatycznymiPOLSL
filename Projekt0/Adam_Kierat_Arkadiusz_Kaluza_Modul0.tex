%%%%%%%%%%%%  Generated using docx2latex.com  %%%%%%%%%%%%%%

%%%%%%%%%%%%  v2.0.0-beta  %%%%%%%%%%%%%%

\documentclass[12pt]{article}
\usepackage{amsmath}
\usepackage{latexsym}
\usepackage{amsfonts}
\usepackage[normalem]{ulem}
\usepackage{soul}
\usepackage{array}
\usepackage{amssymb}
\usepackage{extarrows}
\usepackage{graphicx}
\usepackage[backend=biber,
style=numeric,
sorting=none,
isbn=false,
doi=false,
url=false,
]{biblatex}\addbibresource{bibliography.bib}

\usepackage{subfig}
\usepackage{wrapfig}
\usepackage{wasysym}
\usepackage{enumitem}
\usepackage{adjustbox}
\usepackage{ragged2e}
\usepackage[svgnames,table]{xcolor}
\usepackage{tikz}
\usepackage{longtable}
\usepackage{changepage}
\usepackage{setspace}
\usepackage{hhline}
\usepackage{multicol}
\usepackage{tabto}
\usepackage{float}
\usepackage{multirow}
\usepackage{makecell}
\usepackage{fancyhdr}
\usepackage[toc,page]{appendix}
\usepackage[hidelinks]{hyperref}
\usetikzlibrary{shapes.symbols,shapes.geometric,shadows,arrows.meta}
\tikzset{>={Latex[width=1.5mm,length=2mm]}}
\usepackage{flowchart}\usepackage[paperheight=11.69in,paperwidth=8.27in,left=1.0in,right=1.0in,top=1.0in,bottom=1.0in,headheight=1in]{geometry}
\usepackage[utf8]{inputenc}
\usepackage[T1]{fontenc}
\TabPositions{0.5in,1.0in,1.5in,2.0in,2.5in,3.0in,3.5in,4.0in,4.5in,5.0in,5.5in,6.0in,}

\urlstyle{same}

\renewcommand{\_}{\kern-1.5pt\textunderscore\kern-1.5pt}

 %%%%%%%%%%%%  Set Depths for Sections  %%%%%%%%%%%%%%

% 1) Section
% 1.1) SubSection
% 1.1.1) SubSubSection
% 1.1.1.1) Paragraph
% 1.1.1.1.1) Subparagraph


\setcounter{tocdepth}{5}
\setcounter{secnumdepth}{5}


 %%%%%%%%%%%%  Set Depths for Nested Lists created by \begin{enumerate}  %%%%%%%%%%%%%%


\setlistdepth{9}
\renewlist{enumerate}{enumerate}{9}
		\setlist[enumerate,1]{label=\arabic*)}
		\setlist[enumerate,2]{label=\alph*)}
		\setlist[enumerate,3]{label=(\roman*)}
		\setlist[enumerate,4]{label=(\arabic*)}
		\setlist[enumerate,5]{label=(\Alph*)}
		\setlist[enumerate,6]{label=(\Roman*)}
		\setlist[enumerate,7]{label=\arabic*}
		\setlist[enumerate,8]{label=\alph*}
		\setlist[enumerate,9]{label=\roman*}

\renewlist{itemize}{itemize}{9}
		\setlist[itemize]{label=$\cdot$}
		\setlist[itemize,1]{label=\textbullet}
		\setlist[itemize,2]{label=$\circ$}
		\setlist[itemize,3]{label=$\ast$}
		\setlist[itemize,4]{label=$\dagger$}
		\setlist[itemize,5]{label=$\triangleright$}
		\setlist[itemize,6]{label=$\bigstar$}
		\setlist[itemize,7]{label=$\blacklozenge$}
		\setlist[itemize,8]{label=$\prime$}

\setlength{\topsep}{0pt}\setlength{\parindent}{0pt}

 %%%%%%%%%%%%  This sets linespacing (verticle gap between Lines) Default=1 %%%%%%%%%%%%%%


\renewcommand{\arraystretch}{1.3}


%%%%%%%%%%%%%%%%%%%% Document code starts here %%%%%%%%%%%%%%%%%%%%



\begin{document}
\textbf{Dokument roboczy projekt ZSI Moduł 0}\par


\vspace{\baselineskip}
\textbf{Arkadiusz Kałuża, Adam Kierat}\par


\vspace{\baselineskip}
Przygotowanie dokumentu roboczego zawierającego zapisy dotyczące punktów wyróżnionych w README na platformie zdalnej edukacji.\par


\vspace{\baselineskip}
\begin{itemize}
	\item \textbf{A1. Ustanowienie współdzielonego repozytorium}\\
Ustanowienie\ współdzielonego\ repozytorium\ na\ platformie\ github.com\ \ \ \ \ \ \ \ \ \ \ \ \ \ \ \       Link: \href{https://github.com/arcziko9/ZarzadzanieSystemamiInformatycznymiPOLSL}{\textcolor[HTML]{1155CC}{\ul{https://github.com/arcziko9/ZarzadzanieSystemamiInformatycznymiPOLSL}\\
}}\par

	\item \textbf{B1. Ustalenie z prowadzącym tematu projektu.}\\
Prowadzący zajęcia przydzielił nam projekt numer 8:\\
\\
Bazując na liście 15 przykładów rozwiązań SaaS (Software as a Service) dostępnej pod adresem [\href{https://joshfechter.com/software-service-examples}{\textcolor[HTML]{1155CC}{\ul{https://joshfechter.com/software-service-examples}}}], dokonaj wyboru jednej usługi, przeprowadź (w miarę możliwości) jej testową eksploatację i jako rezultat realizacji zadania przedstaw krótką charakterystykę wraz prezentacją nie więcej niż dwóch słabych stron oraz nie więcej niż dwóch silnych stron).\\
\par

	\item \textbf{B3. Przygotowanie notatki z ASD: 3 pozycje, 3 najważniejsze wyróżniki, linki do witryn zawierających szczegóły.\\
}Office 365 jest pakietem biurowym, wyróżnia go to, że obsługa danych odbywa się w chmurze. Otwiera nam to całkiem nowy sposób do pracy z dokumentami.
\end{itemize}\par

\begin{adjustwidth}{0.5in}{0.0in}
Dzięki temu, że nasze $``$biuro$"$  jest obsługiwane w chmurze możemy uzyskać dostęp do swoich aplikacji i plików z dowolnego urządzenia np. Komputer PC, Komputer MAC, tablety oraz urządzeń mobilnych.\par

\end{adjustwidth}

\begin{adjustwidth}{0.5in}{0.0in}
Strony na których możemy znaleźć informacje o usłudze Office 365:\\
\href{https://ujeb.se/WmG7v}{\textcolor[HTML]{1155CC}{\ul{https://ujeb.se/WmG7v}\\
\href{https://ujeb.se/x767W}{https://ujeb.se/x767W}}}\par

\end{adjustwidth}

\begin{adjustwidth}{0.5in}{0.0in}
\href{https://ujeb.se/rQmjf}{\textcolor[HTML]{1155CC}{\ul{https://ujeb.se/rQmjf}}} \\
\par

\end{adjustwidth}

\begin{itemize}
	\item \textbf{C1.Opracowanie koncepcji realizacji projektu.}\par

\begin{itemize}
	\item Silne strony:\par

\begin{itemize}
	\item Największą zaletą Office 365 jest możliwość pracy tak naprawdę z dowolnego miejsca i urządzenia. Jedynym wymogiem jest połączenie z Internetem ponieważ usługa jest całkowicie oparta na chmurze. Można uzyskiwać dostęp do poczty e-mail i programów pakietu Office takich jak Word, PowerPoint, Excel itd. Jest to świetne rozwiązanie dla firm w których zatrudnieni są pracownicy, którzy pracują zdalnie lub mają wiele podróży służbowych. Usługa świetnie sprawdzi się też gdy firma posiada wiele lokalizacji.\par

	\item Kolejną silną stroną jest prostsza współpraca między pracownikami. Jeśli w twojej firmie istnieje potrzeba współtworzenia, lub edytowania dokumentu przez wiele osób to Office 365 pozwala nam na pracowanie nad tą samą wersją dokumentu, a zmiany będziemy otrzymywać w czasie rzeczywistym. Użytkownik może również udostępniać dostęp do swoich plików, więc pracownik nie musi wysyłać swoich plików jako załączników. Dzięki współpracownicy mogą pracować na jednym pliku, a nie mieć wielu kopii. Istnieje przechowywanie wersji, więc istnieje możliwość powrotu do starszej wersji pliku.\par


\end{itemize}
	\item Słabe strony:\par

\begin{itemize}
	\item Wersja Office Word w przeglądarce nie pozwala użytkownikom na dostęp do dokumentów chronionych hasłem, ani nie może uruchamiać makr wewnątrz dokumentu podczas pracy online.\par


\vspace{\baselineskip}
	\item Microsoft dokłada wszelkich starań, aby zapewnić ochronę chmury. Nie oznacza to jednak, że nie ma wad, które mogłyby zagrozić Twojemu bezpieczeństwu. Na przykład w tym głośnym wystąpieniu na początku tego roku zespół badawczy Google Project Zero zajmujący się bezpieczeństwem odkrył obawy związane z bezpieczeństwem w Edge, innym produkcie Microsoft. Według naukowców błąd pozwala atakującym kontrolować informacje zapisane w pamięci aplikacji, co może mieć katastrofalne skutki dla bezpieczeństwa danych. Kolejny krytyczny błąd, zgłoszony zaledwie tydzień przed wiadomością o usterce zabezpieczeń Edge, ujawnił potencjalnie wrażliwe dane przechowywane na serwerach Windows. Wydanie poprawki dla tej usterki zajęło firmie Microsoft kilka miesięcy. Chociaż Edge nie jest dołączony do pakietu Microsoft Office 365, warto rozważyć te zagrożenia, rozważając przejście na aplikacje chmurowe.
\end{itemize}
\end{itemize}
\end{itemize}\par


\vspace{\baselineskip}

\vspace{\baselineskip}

\printbibliography
\end{document}