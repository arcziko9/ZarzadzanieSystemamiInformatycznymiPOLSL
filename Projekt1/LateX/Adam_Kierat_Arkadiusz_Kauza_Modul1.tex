%%%%%%%%%%%%  Generated using docx2latex.com  %%%%%%%%%%%%%%

%%%%%%%%%%%%  v2.0.0-beta  %%%%%%%%%%%%%%

\documentclass[12pt]{article}
\usepackage{amsmath}
\usepackage{latexsym}
\usepackage{amsfonts}
\usepackage[normalem]{ulem}
\usepackage{soul}
\usepackage{array}
\usepackage{amssymb}
\usepackage{extarrows}
\usepackage{graphicx}
\usepackage[backend=biber,
style=numeric,
sorting=none,
isbn=false,
doi=false,
url=false,
]{biblatex}\addbibresource{bibliography.bib}

\usepackage{subfig}
\usepackage{wrapfig}
\usepackage{wasysym}
\usepackage{enumitem}
\usepackage{adjustbox}
\usepackage{ragged2e}
\usepackage[svgnames,table]{xcolor}
\usepackage{tikz}
\usepackage{longtable}
\usepackage{changepage}
\usepackage{setspace}
\usepackage{hhline}
\usepackage{multicol}
\usepackage{tabto}
\usepackage{float}
\usepackage{multirow}
\usepackage{makecell}
\usepackage{fancyhdr}
\usepackage[toc,page]{appendix}
\usepackage[hidelinks]{hyperref}
\usetikzlibrary{shapes.symbols,shapes.geometric,shadows,arrows.meta}
\tikzset{>={Latex[width=1.5mm,length=2mm]}}
\usepackage{flowchart}\usepackage[paperheight=11.69in,paperwidth=8.27in,left=1.0in,right=1.0in,top=1.0in,bottom=1.0in,headheight=1in]{geometry}
\usepackage[utf8]{inputenc}
\usepackage[T1]{fontenc}
\TabPositions{0.5in,1.0in,1.5in,2.0in,2.5in,3.0in,3.5in,4.0in,4.5in,5.0in,5.5in,6.0in,}

\urlstyle{same}

\renewcommand{\_}{\kern-1.5pt\textunderscore\kern-1.5pt}

 %%%%%%%%%%%%  Set Depths for Sections  %%%%%%%%%%%%%%

% 1) Section
% 1.1) SubSection
% 1.1.1) SubSubSection
% 1.1.1.1) Paragraph
% 1.1.1.1.1) Subparagraph


\setcounter{tocdepth}{5}
\setcounter{secnumdepth}{5}


 %%%%%%%%%%%%  Set Depths for Nested Lists created by \begin{enumerate}  %%%%%%%%%%%%%%


\setlistdepth{9}
\renewlist{enumerate}{enumerate}{9}
		\setlist[enumerate,1]{label=\arabic*)}
		\setlist[enumerate,2]{label=\alph*)}
		\setlist[enumerate,3]{label=(\roman*)}
		\setlist[enumerate,4]{label=(\arabic*)}
		\setlist[enumerate,5]{label=(\Alph*)}
		\setlist[enumerate,6]{label=(\Roman*)}
		\setlist[enumerate,7]{label=\arabic*}
		\setlist[enumerate,8]{label=\alph*}
		\setlist[enumerate,9]{label=\roman*}

\renewlist{itemize}{itemize}{9}
		\setlist[itemize]{label=$\cdot$}
		\setlist[itemize,1]{label=\textbullet}
		\setlist[itemize,2]{label=$\circ$}
		\setlist[itemize,3]{label=$\ast$}
		\setlist[itemize,4]{label=$\dagger$}
		\setlist[itemize,5]{label=$\triangleright$}
		\setlist[itemize,6]{label=$\bigstar$}
		\setlist[itemize,7]{label=$\blacklozenge$}
		\setlist[itemize,8]{label=$\prime$}

\setlength{\topsep}{0pt}\setlength{\parindent}{0pt}

 %%%%%%%%%%%%  This sets linespacing (verticle gap between Lines) Default=1 %%%%%%%%%%%%%%


\renewcommand{\arraystretch}{1.3}


%%%%%%%%%%%%%%%%%%%% Document code starts here %%%%%%%%%%%%%%%%%%%%



\begin{document}
\textbf{Dokument roboczy projekt ZSI Moduł 1}\par


\vspace{\baselineskip}
\textbf{Arkadiusz Kałuża, Adam Kierat}\par


\vspace{\baselineskip}
Przygotowanie dokumentu roboczego zawierającego zapisy dotyczące punktów wyróżnionych we wzorze dokumentu roboczego dostęnego na platformie zdalnej edukacji.\par


\vspace{\baselineskip}
\begin{itemize}
	\item \textbf{A1. Ustanowienie współdzielonego repozytorium}\\
Ustanowienie\ współdzielonego\ repozytorium\ na\ platformie\ github.com\ \ \ \ \ \ \ \ \ \ \ \ \ \ \ \       Link: \href{https://github.com/arcziko9/ZarzadzanieSystemamiInformatycznymiPOLSL}{\textcolor[HTML]{1155CC}{\ul{https://github.com/arcziko9/ZarzadzanieSystemamiInformatycznymiPOLSL}}}\\
\par

	\item \textbf{B1. Ustalenie z prowadzącym tematu projektu.}\\
Sekcja wybrała projekt numer 1.7:\\
 \tab a. Bezobsługowa instalacja aplikacji w systemach Microsoft Windows\\
 \tab b. Zarządzanie infrastrukturą sieciową\\
 \tab c. Podpis elektroniczny\\
\par

	\item \textbf{B3. Przygotowanie notatki z ASD: 3 pozycje, 3 najważniejsze wyróżniki, linki do witryn zawierających szczegóły.}\par

\begin{itemize}
	\item a. Systemy z rodziny Microsoft, a mianowicie Windows Server oferuje nam bezobsługową instalację, aplikacji. Jest to bardzo przydatne rozwiązanie w warunkach biznesowych, gdy do obsługi mamy wiele komputerów. Pozwala to zaoszczędzić wiele czasu i pieniędzy.\\
Przydatne strony zawierające informację o tej czynności:\par

\begin{itemize}
	\item \href{https://ujeb.se/N2AF6}{https://ujeb.se/N2AF6}\par

	\item \href{https://ujeb.se/jVXwj}{https://ujeb.se/jVXwj}\par

	\item \href{https://ujeb.se/295Vy}{https://ujeb.se/295Vy}\par


\end{itemize}
	\item b.\ Zarządządzanie infrastrukturą sieciową wiąże się z monitorowaniem zarówno parametrów sieci, urządzeń sieciowych, itp., jak również działania i integralności  \tab serwerów oraz usług na nich uruchomionych.\\
 Przydatne strony linki:\par

\begin{itemize}
	\item \href{https://www.zabbix.com/}{https://www.zabbix.com/}\par

	\item \href{https://www.activexperts.com/}{https://www.activexperts.com/}\par

	\item \href{https://www.nagios.org/}{https://www.nagios.org/}\par


\end{itemize}
	\item c. Podpis elektroniczny to połączenie danych w formie elektronicznej oraz innych, które razem tworzą możliwość identyfikacji osoby, która taki podpis złożyła. Umożliwia on identyfikację podmiotów, które przesyłają dokumenty drogą elektroniczną.\par

\begin{itemize}
	\item \href{https://ujeb.se/pCsuT}{https://ujeb.se/pCsuT}\par

	\item \href{https://ujeb.se/k1wLL}{https://ujeb.se/k1wLL}\\

\end{itemize}
\end{itemize}
\end{itemize}\par

\begin{itemize}
	\item \textbf{C1.Opracowanie koncepcji realizacji projektu.}\par

\begin{itemize}
	\item Aby umożliwic instalowanie programów zdalnie na komputerach z systemami z rodziny Windows potrzebujemy mieć zainstalowaną na naszym serwrze domenę Active Directory. Warto wspomnieć, że istnieją dwie metody zrealizowania tej cyznności pierwszą z nich jest \textbf{Przypisywanie oprogramowania} w której program możemy przypisać do użytkownika i zostanie on zainstalowany podczas logowania użytkownika do komputera, tak samo można przypisać program do do komputera wtedy zostanie on zainstalowany podczas uruchomienia komputera i będzie dosępny dla wszystkich użytkowników, którzy logują się na tym kompuerze.\\
Drugim sposobem jest \textbf{Publikowanie oprogramowania} w tym przypadku administator publikuje dla użytkowników program, który jest wyświetlany w oknie dialogowyn \textbf{Dodaj lub usuń programy} i możę być stamtąd zainstalowany. W jednym z powyższych linków jest szczegółowa instrukcja krok po kroku, która przeprowadza przez etapy zdalnego instalowania oprogramowania.\par

	\item Dobrym przykłdem oprogramowania moniturującym szereg parametrów sieci, jak również działania i integralności serwerów jest \textbf{Zabbix}. Jest to otwarte rozwiązanie klasy biznesowej stworzone do powyższych czynności. Zabbix używa elastycznego moechanizmu powiadomień, pozwalającego użytkownikom skonfigurować powiadomenia e-mail dla praktycznie każdego zdarzenia. Pozwala to na szybkąreakację na problemu z serwerami. Zabbix oferuje doskonałe opcje raportowania i wizualizacji zebranych danych. To czyni Zabbix idealnym do planowania zdolności. Zabbix wspiera zarówno przekazywanie (polling) jak i przechwytywanie (trapping) danych. Wszystkie raporty i statystyki Zabbix, jak również parametry konfiguracyjne, są dostępne z poziomu interfejsu użytkownika bazującego na stronie www. Taki interfejs użytkownika zapewnia, że status sieci i stan serwerów jest dostępny z dowolnego miejsca. Prawidłowo skonfigurowany Zabbix może grać ważną rolę w monitorowaniu infrastruktury IT. Jest to fakt zarówno w przypadku małych organizacji z kilkoma serwerami, jak również w przypadku dużych firm w wieloma serwerami. \\
Możliwości Zabbix:\par

\begin{itemize}
	\item \textbf{Zbieranie danych}\par

\begin{itemize}
	\item kontrole dostępności i wydajności\par

	\item wsparcie dla SNMP (trapping, polling), IPMI, JMX, monitorowania Vmware\par

	\item kontrole użytkownika\par

	\item zbieranie danych w dowolnych, ustalanych odstępach czasu\par

	\item wykonywane przez serwer/proxy i przez agentów\par


\end{itemize}
	\item \textbf{Elastyczne definicje progowe}\par

\begin{itemize}
	\item można zdefiniować bardzo elastyczne warunki progowe dla problemu, zwane wyzwalaczami, korzystające z wartości z bazy danych\par


\end{itemize}
	\item \textbf{Wysoce konfigurowalne alarmowanie}\par

\begin{itemize}
	\item wysyłanie powiadomień może być modyfikowane w zakresie harmonogramu eskalacji, odbiorców, typów mediów\par

	\item powiadomienia mogą być przygotowywane i opracowywane z użyciem zmiennych makr\par

	\item automatyzacja akcji włącznie ze zdalnymi komendami\par


\end{itemize}
	\item \textbf{Wykresy w czasie rzeczywistym}\par

\begin{itemize}
	\item monitorowane pozycje są wykreślane w czasie rzeczywistym przy pomocy wbudowanych funkcji graficznych\par


\end{itemize}
	\item \textbf{Możliwość monitorowania stron www}\par

\begin{itemize}
	\item Zabbix potrafi symulować kliknięcia myszy na stronach wwww i sprawdzać ich funkcjonowanie i czas odpowiedzi\par


\end{itemize}
	\item \textbf{Rozbudowane opcje wizualizacji}\par

\begin{itemize}
	\item możliwość tworzenia własnych wykresów kumulujących wiele pozycji w jeden widok\par

	\item mapy sieci\par

	\item własne ekrany i pokazy slajdów do zastosowania na tablicy\par

	\item raporty\par

	\item podgląd wysokiego-poziomu (biznesowy) monitorowanych zasobów\par


\end{itemize}
	\item \textbf{Magazyn danych historycznych}\par

\begin{itemize}
	\item Dane przechowywane w bazie\par

	\item Konfigurwalna historia\par

	\item Wbudowane prodecury porządkowania\par


\end{itemize}
	\item \textbf{Prosta konfiguracja}\par

\setlength{\parskip}{5.04pt}
\begin{itemize}
	\item dodawanie monitorowanych urządzeń jako hosty\par

	\item hosty ustawiane są na monitorowanie zaraz po znalezieniu się w bazie\par

	\item wykorzystywanie wzorców do ustawiania monitorowanych urządzeń\par


\end{itemize}
	\item \textbf{Użycie wzorców}\par

\begin{itemize}
	\item grupowanie kontroli we wzorcach\par

	\item wzorce mogą zawierać inne wzorce\par


\end{itemize}
	\item \textbf{Wykrywanie sieci}\par

\begin{itemize}
	\item automatyczne wykrywanie urządzeń sieciowych\par

	\item automatyczna rejestracja agentów\par

	\item wykrywanie systemów plików, interfejsów sieciowych i identyfikatorów OID SNMP\par


\end{itemize}
	\item \textbf{Szybki interfejs www}\par

\begin{itemize}
	\item interfejs użytkownika bazujący na stronach www w PHP\par

	\item dostępny z dowolnej lokalizacji\par

	\item można zrobić nim wszystko\par

	\item logi audytu\par


\end{itemize}
	\item \textbf{API Zabbix}\par

\begin{itemize}
	\item API Zabbix udostępnia interfejs programowy Zabbixa do masowej manipulacji, integracji z oprogramowaniem firm trzecich oraz innych celów\par


\end{itemize}
	\item \textbf{System usprawnień}\par

\begin{itemize}
	\item bezpieczna autoryzacja użytkowników\par

	\item niektórzy użytkownicy mogą być ograniczeni do niektórych widoków\par


\end{itemize}
	\item \textbf{W pełni wyposażony i łątwo rozszerzalny agent}\par

\begin{itemize}
	\item zainstalowany na monitorowanych urządzeniach\par

	\item może działać zarówno na systemie Linux jak i na Windows\par


\vspace{\baselineskip}

\vspace{\baselineskip}

\vspace{\baselineskip}

\vspace{\baselineskip}

\end{itemize}
\end{itemize}
	\item \textbf{Jak uzyskać kwalifikowany podpis elektroniczny?}
\end{itemize}
\end{itemize}\par

Kwalifikowany podpis elektroniczny jest narzędziem komercyjnym, można go kupić u certyfikowanych dostawców, nadzorowanych przez Ministerstwo Cyfryzacji. Listę dostawców można znaleźć na stronie Narodowego Centrum Certyfikacji (NCCert).\par


\vspace{\baselineskip}
Wysokość opłaty ustalana jest przez podmioty oferujące podpis elektroniczny - cena zależy od długości ważności certyfikatu (rok lub dwa lata) oraz rodzaju urządzenia do składania podpisu elektronicznego (czytnik kart USB, token USB lub PCMCIA).\par


\vspace{\baselineskip}
\textbf{Kupując kwalifikowany podpis elektroniczny, należy:}\par


\vspace{\baselineskip}
\begin{itemize}
	\item podpisać z dostawcą tzw. umowę subskrypcyjną - jej warunki określa kodeks postępowania certyfikacyjnego lub polityka certyfikacji,\par


\vspace{\baselineskip}
	\item stawić się osobiście w centrum certyfikacji z dokumentem tożsamości - aby dostawca mógł potwierdzić tożsamość kupującego,\par


\vspace{\baselineskip}
	\item zainstalować otrzymane oprogramowanie.
\end{itemize}\par


\vspace{\baselineskip}
\textbf{Jak podpisać dokument podpisem elektronicznym?}\par


\vspace{\baselineskip}
Gdy\ już zakupimy zestaw do podpisu elektronicznego (oprogramowanie, kartę kryptograficzną i czytnik kart)  u jednego z certyfikowanych dostawców, w pierwszym kroku należy zainstalować otrzymane oprogramowanie na komputerze.\par

Następnie należy włożyć kartę do czytnika kart oraz uruchomić oprogramowanie.\par

Kolejnym krokiem jest użycie przycisku $``$podpisz$"$  oraz wybór dokumentu do podpisania z komputera.\par

Po wybraniu odpowiednich dokumentów wystarczy wpisać PIN w żądane miejsce.\par


\vspace{\baselineskip}
\textbf{Jakie sprawy można załatwić z wykorzystaniem podpisu elektronicznego?}\par


\vspace{\baselineskip}
Wykorzystując bezpieczny podpis elektroniczny, można podpisać plik dokumentów, np.:\par


\vspace{\baselineskip}
\begin{itemize}
	\item faktury elektroniczne,\par


\vspace{\baselineskip}
	\item umowy handlowe,\par


\vspace{\baselineskip}
	\item zarejestrować działalność gospodarczą,\par


\vspace{\baselineskip}
	\item pisma procesowe w sądowym postępowaniu upominawczym,\par


\vspace{\baselineskip}
	\item e-deklaracje podatkowe,
\end{itemize}\par


\vspace{\baselineskip}

\vspace{\baselineskip}

\printbibliography
\end{document}